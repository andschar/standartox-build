%% table that compares different ecotoxicological databases

%%%% CONTINUE HERE

\begin{sidewaystable}
% \begin{table}
\caption{Different databases that provide ecotoxicological data. Abbreviations: \textbf{ALL:} Most important parameters, such as chemical, Organism, Duration for filtering ecotoxicological data are incorporated. \textbf{WEB:} Accessible via a web application and a graphical user interface, not providing programmatic access.}
\label{tab:database-differences}
\begin{tabular}{|m{3cm}|m{3cm}|m{2cm}|m{2cm}|m{1cm}|l|}
\hline
Database & Publisher & Filter & Aggregation, Selection & Access & website \\
\hline
Comptox & Environmental Protection Agency & Chemical & no & WEB, file & https://comptox.epa.gov/dashboard \\
\hline
Ecotox & Environmental Protection Agency & ALL & no & WEB, file & https://webetox.uba.de/webETOX/index.do \\
\hline
EnviroTox & Health and Environmental Sciences Institute & ALL & chemical, organism & WEB & https://envirotoxdatabase.org \\
\hline
Etox & Umweltbundesamt & ALL & no & WEB & https://webetox.uba.de/webETOX/index.do \\
\hline
Pesticide Property Data Base (PPDB) & University of Hertfordshire & fixed values & manual selection & WEB, file & https://sitem.herts.ac.uk/aeru/ppdb/index.htm \\
\hline
\textbf{Standartox} & \textbf{this article} & \textbf{ALL} & \textbf{chemical, organism} & \textbf{API, WEB} & \textbf{http://standartox.uni-landau.de} \\
\hline
\end{tabular}
% \end{table}
\end{sidewaystable}


