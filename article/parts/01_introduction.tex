\section*{Introduction}
An increasing number of chemicals such as pharmaceuticals, pesticides and synthetic hormones are in daily use all over the world. In Europe alone, some 100,000 chemicals are estimated to be in current use, whereof 30,000 are produced in quantities larger than one ton per year \citep{breithaupt_costs_2006}. Except for pesticides that are released into the environment deliberately, most chemicals enter the environment as a result of their use through different paths (e.g. atmospheric emission and deposition or discharge through wastewater) \citep{schwarzenbach_challenge_2006}. In the environment, chemicals can adversely affect populations and communities and in turn related ecosystem functions \citep{rhind_anthropogenic_2009, clements_community_2009, hallmann_declines_2014, barracaracciolo_pharmaceuticals_2015, johnston_review_2015, hallmann_more_2017, sanchez-bayo_worldwide_2019}. Ultimately, this may compromise natures contribution to human well-being, for example the ecosystem services clean drinking and irrigation water, food and climate regulation \citep{peters_review_2013, vandersluijs_neonicotinoids_2013, barracaracciolo_pharmaceuticals_2015, yamamuro_neonicotinoids_2019}. Pollution with man-made chemicals has been identified as one of three major environmental problems for which research gaps hamper the derivation of planetary boundaries, i.e. thresholds beyond which irreversible state shifts may occur \citep{steffen_anthropocene_2007, steffen_planetary_2015}. \citet{bernhardt_synthetic_2017} argue that the knowledge gap how chemicals affect populations, communities and in turn ecosystem functions and services, may impede the accomplishment of the Sustainable Development Goals of the United Nations. Even highly regulated chemical compounds such as pesticides have been shown to cause strong adverse effects on organisms, such as birds \citep{hallmann_declines_2014} or fish \citep{yamamuro_neonicotinoids_2019}, questioning the current regulation efforts \citep{schafer_future_2019}. To evaluate the risks from chemicals for ecosystems data on their toxicity, which is typically produced in standardised ecotoxicological laboratory tests is required. For example, \citet{morrissey_neonicotinoid_2015} used ecotoxicological test results from 49 insect and crustacean species to evaluate the effect of neonicotinoids, an insecticide class on the aquatic ecosystem. Also, \citet{malaj_organic_2014} compiled inter alia experimental toxicity test results for 223 various chemical compounds to assess the health of freshwater ecosystems in Europe. Similarly, permissible environmental concentrations are often derived from these test data, typically by a combination with safety factors to account for uncertainties. Given that data sets with large numbers of analysed chemicals are becoming increasingly available, ecotoxicity data are pivotal for regulatory risk assessment and science. To date, only few initiatives exist that aim to create a public resource of harmonized and quality controlled ecotoxicological data, such as the United States Environmental Protection Agency ECOTOXicology knowledgebase (EPA ECOTOX) \citep{elonen_ecotoxicology_2018}, the German Environmental Agency's Information System Ecotoxicology and Environmental Quality Targets (ETOX) \citep{umweltbundesamt_etox_2019} or the Pesticide Property Data Base (PPDB) \citep{lewis_international_2016}. The former two compile all available test results into a database. However, for many chemicals, multiple ecotoxicity values are available for the same test organism, which can vary strongly. Hence, these databases can contain multiple entries for the same test without quality information and with different units, hampering reproducible science if different values are selected for studies. The PPDB database is confined to pesticides and provides harmonised and quality controlled data for a few selected test organisms, thereby ignoring vast amounts of data for other species and chemicals. Moreover, data analyses often require links to additional data resources, for example to append additional chemical and species information (e.g. chemical properties, habitat of species), which calls for more automated procedures.\\

Standartox makes use of the publicly available and quarterly released EPA ECOTOX database which comprises almost 1,000,000 ecotoxicological test results from almost 12,000 chemicals and 13,000 taxa \citep{usepa_ecotox_2019}.


We developed Standartox, a tool and database that aims to overcome the limitations of other databases by continuously incorporating the ever-growing number of test results in an automated process workflow that ultimately leads to single aggregated data point for a specific chemical-organism test combination, representing the toxicity of a chemical. Hence, Standartox provides the basis for reproducible science and combines information from different sources to simplify the derivation of more advanced risk indicators such as Species Sensitivity Distributions (SSD) and Toxic Units (TU), which represent two prominent concepts to assess effects on organisms in ecotoxicology \citep{posthuma_species_2002, kefford_definition_2011, schafer_effects_2011}. Besides aggregating ecotoxicological test results, Standartox provides a concise overview of the status of tested chemicals, allowing the identification of potential research gaps. Standartox comes with two front-ends,  a web application (APP) and a R \citep{rcoreteam_language_2017} package, which accesses a application programming interface (API) providing convenience structures and thereby largely reducing processing time for users.

Standartox is needed due to variability of test results:

125 distinct duration units




%%%%%%%%%%%%%%%%%%%%%%%%%%%%%%%%%%%%%%%%%% IDEAS %%%%%%%%%%%%%%%%%%%%%%%%%%%%%%%%%%%%%%%%%
\iffalse



\item Put in citation \citep{hartung_chemical_2009} ? which claims that REACH won't meet their assumptions and say 54 million vertebrate animals are needed for tests that would cost €9.5 billion over the next ten years. I.e 20 times more animals, 6 times the costs in comparison to the official estimates 


For dicussion:
Showed general decline in biodiversity and biomass, but no or not much causes: \citep{butchart_global_2010, hallmann_more_2017, seibold_arthropod_2019} 

Direct effect of chemicals:
include \citep{yamamuro_neonicotinoids_2019} % ecosystem services, food webs

- gerneral stuff about small water bodies: \citep{riley_small_2018}


\item standardized toxicity values
    
\item reproducible research % TODO mention much more often!
    
\item time saving
\fi    

%%%%%%%%%%%%%%%%%%%%%%%%%%%%%%%%%%%%%%%%%% OLD %%%%%%%%%%%%%%%%%%%%%%%%%%%%%%%%%%%%%%%%%%%
\iffalse

A large number of chemicals such as pharmaceuticals, pesticides and synthetic hormones are in daily use all over the world with insufficient knowledge about possible adverse effects on the environment. In Europe alone some 100,000 chemicals are estimated to be in current use, whereof 30,000 are produced in quantities larger than one ton per year \citep{breithaupt_costs_2006}. Chemicals are brought to the environment deliberately - as it is the case of pesticides, or arrive therein as a byproduct from other processes (e.g. atmospheric emissions or wastewater) \citep{schwarzenbach_challenge_2006}. Ecosystems provide essential services to human societies such as drinking and irrigation water, food and climate regulation. These services are products of different ecosystem functions that crucially depend on the integrity of the populations and communities that drive these ecosystem functions. However, besides habitat degradation, climate change and nutrient enrichment, pollution with man-made chemical toxicants threatens these populations and communities in various ways which are currently not fully understood \citep{steffen_anthropocene_2007}. Pollution with man-made chemical toxicants was indeed identified as one of three major environmental problems for which research gaps hamper the derivation of planetary boundaries, i.e. thresholds beyond which irreversible state shifts may occur \citep{steffen_anthropocene_2007}. Bernhardt et al. \citet{bernhardt_synthetic_2017} argue further that the knowledge gap how chemicals effect populations and communities and hence ecosystem functions and ecosystem services, would also impede society's ability to accomplish the Sustainable Development Goals of the United Nations. According to Breithaupt \citet{breithaupt_costs_2006} less than one percent of chemicals released to the environment are thoroughly tested. Although for some chemicals advanced and standardized \citep{oecd_oecd_2018} test methods have been developed, ecotoxicological tests for a lot of chemicals, especially newly emerging ones remain scarce and if existent, the information is represented sparsely and in inconsistent formats. \citep{gessner_fostering_2016}. Therefore we developed a data base that collects, processes and aggregates ecotoxicological tests results in order to subsequently publish it in an harmonized form as a web application - the \etoxbase{} tool: \href{http://139.14.20.252:3838/etox-base-shiny/}. This should facilitate researchers the access to ecotoxicological test results and support modern chemical risk assessment (CRA) workflows. The \etoxbase{} makes use of ecotoxicological test data, obtained from the ECOTOXicology knowledgebase (ECOTOX) created by the U.S. Environmental Protection Agency (EPA). ECOTOX collects raw, non harmonized ecotoxicological test data on aquatic and terrestrial wildlife as well as plants and publishes it on a quarterly basis. By the time of writing it contained 926,108 test results, comprising 11,685 chemicals tested on 12,668 different species \citep{elonen_ecotoxicology_2018}. The collected ecotoxicological test data contain a variety of measured endpoints such as Effective Concentrations (\ecfifty{}) values, No-observed effect concentrations (NOEC) or lowest observed effect concentrations (LOEC). Likewise the data set contains different effect measures such as mortality, intoxication and growth as well as a multiplicity of test durations ranging from seconds to weeks or even years. The \etoxbase{} aggregates this information according to the user's inputs. These can then be used for the derivation of risk indicators such as Species Sensitivity Distributions (SSDs) \citep{posthuma_species_2002} and Toxic Units (TUs), which represent two prominent concepts to assess effects on organisms in ecotoxicology. The former combines toxicity data of several organism groups towards a chemicals to estimate effects on biotic communities, the latter refers to effects of a specific organism (group) towards a chemical. The two concepts are widely used in ecotoxicology \citep{kefford_definition_2011, schafer_effects_2011} since the allow for a comparison of toxicities across multiple chemicals and biological communities, respectively. Although performed on one and the same organism and chemical, outcomes of different ecotoxicological tests vary greatly due to differences in test parameters, genetic differences in individual populations or other irrepressible factors making a selection of appropriate test results for CRA laborious. The \etoxbase{} aims to relief this process and provide single toxicity values for organism, chemical and test duration combinations. The \etoxbase{} downloads every new version of the ECOTOX data base and performs several quality checks, e.g. unit harmonization, and cleaning steps on it. Hence, new scientific test results are constantly incorporated. Additionally the \etoxbase{} queries chemical- and organism-specific variables from other open data bases to fill information gaps such as water solubility, organism habitat or regional occurrence patterns in the ECOTOX data base.

\fi

