%%The Methods should include detailed text describing any steps or procedures 
%%used in producing the data, including full descriptions of the experimental 
%%design, data acquisition assays, and any computational processing (e.g. 
%%normalization, image feature extraction). Related methods should be grouped 
%%under corresponding subheadings where possible, and methods should be described 
%%in enough detail to allow other researchers to interpret and repeat, if required, 
%%the full study. Specific data outputs should be explicitly referenced via data 
%%citation (see Data Records and Data Citations, below). Authors should cite 
%%previous descriptions of the methods under use, but ideally the method 
%%descriptions should be complete enough for others to understand and reproduce 
%%the methods and processing steps without referring to associated publications. 
%%There is no limit to the length of the Methods section.
\section*{Methods}

Standartox consists of three parts of software explained in detail below. Firstly a data acquisition and preparation processing pipeline which downloads quarterly released new versions of the EPA ECOTOX, upon which Standartox is built. Secondly, Standartox is accessible via a web application (APP) and an application programming interface (API). Lastly, the R package standartox constitutes an easily usable way of accessing the tool via the API. All the software is written in R \citep{rcoreteam_language_2017} and SQL scripts.

\subsection*{Data acquisition and preparation pipe line}
The EPA ECOTOX, which is released on a quarterly basis (current release: 12.09.2019) is checked for new releases in monthly intervals and downloaded as soon as a new release is published. Subsequently the EPA ECOTOX is built into a PostgreSQL data base and SQL functions for further processing are added and known errors are corrected. Lookup tables for filtering and converting the data are uploaded. Thereafter, a meta data table is created in the data base, comprising the version number of the EPA ECOTOX. In a next step Chemical Abstracts Service (CAS) identifiers are used to query identifier data bases, including the Chemical Identifier Resolver (CIR) service <CITATION> and the Pubchem data base <CITATION>. Chemical identifiers (CAS, InchI) and taxonomic names are then used to query chemical and taxonomic specific data bases (Table \ref{tab:data-base-additional}) to add information to the local data base. The data is aggregated and unique values are derived. <PUT LOOKUPTABLES HERE>. From here on the Standartox data set is compiled, whereupon concentration and duration units are converted (see tables \ref{tab:conv-concentration}, \ref{tab:conv-duration}) in order to harmonize data. A catalog object, listing unique values for categorical variables and ranges for continuous variables, respectively is created. To guarantee the accuracy of the data check scripts are run at the end. Single processing scripts are listed in figure \ref{fig:pipeline-tree}. The final Standartox table together with the catalog are exported. Both the APP and the API access these objects.

\subsection*{Web application (APP) and Application Programming Interface (API)}

The web application and the API load the compressed serialized Standartox data into memory and allow a client to interact with it. The client can then call the functions stx\_filter() and stx\_aggregate() to filter and aggregate the data according to specific parameters (Table \ref{tab:app-parameters}).

\subsubsection*{Web application (APP)}
The interactive APP was built in R, using the shiny web application framework \citep{chang_shiny_2018} which runs with the help of a shiny server \citep{HOW-TO-CITE-SHINY-SERVER} on www.standartox.uni-landau.de. It is set of R functions that allow a user to filter and aggregate the data. The data can then be downloaded.

\subsubsection*{Application programming interface (API)}
The API is built by using the R package plumber \citep{trestletechnologyllc_plumber_2018} which allows for the creation of RESTful APIs from R. Four API-endpoints ('/catalog', '/filter/rds', 'aggregate' and 'meta') that can be queried from the client side were created. The \textit{catalog} API-endpoint returns a JavaScript Object Notation (JSON) file containing a catalog of possible filter parameters to choose from. The \textit{filter} returns the filtered Standartox table as a compressed serialized binary file \citep{klik_fst_2019}, to reduce size and allow for fast user queries. The \textit{aggregate} API-endpoint returns a R function to the client allowing the client the aggregation of the filtered data. Lastly, the \textit{meta} API-endpoint returns a JSON file with meta information, such as the timestamp of the request and the used Standartox version. The API is designed to be used with the R-package standartox and therefore uses serialization methods specific to R (rds() from the R package base and fst() from the package fst).

\begin{figure}
    \includestandalone[scale=0.5]{article/tikz/pipeline-organigram}
    %\input[scale = 0.5]{article/tikz/pipeline-organigram.tex}
    \caption{Organigram of Standartox.}
    \label{fig:stx-organigram}
\end{figure}


\subsubsection*{R-package}
The R-package standartox contains two functions \textit{stx\_catalog()} and \textit{stx\_query()}. The former accesses the \textit{catalog} API-endpoint and returns a R-list of possible filter parameters that can be used in \textit{stx\_query()}. The latter returns a R list of three tables (R data.frames) containing the filtered data set, the aggregated data set and a table with the meta information retrieved from the \textit{meta} API-endpoint.

\subsection*{Code availability}
%%For all studies using custom code in the generation or processing of datasets, 
%%a statement must be included here, indicating whether and how the code can be 
%%accessed, including any restrictions to access. This section should also include 
%%information on the versions of any software used, if relevant, and any specific 
%%variables or parameters used to generate, test, or process the current dataset. 

Standartox's code is distributed on three Github repositories below. All code can freely be accessed under the MIT License <CITATION>. Most of the code is written in R 3.6.1 and associated packages (compare Supplement: \ref{list:r-packages}). Some parts in PostgreSQL 9.6.1.

\begin{itemize}

\item \url{https://github.com/andschar/standartox-build} \newline
Contains the code for the data acquisition and processing pipeline.

\item \url{https://github.com/andschar/standartox-app} \newline
Contains the code for the web application and the API.

\item \url{https://github.com/andschar/standartox} \newline
Contains the code for the R-package.

\end{itemize}

%%%%%%%%%%%%%%%%%%%%%%%%%%%%%%%%%%%%%%%%%% OLD %%%%%%%%%%%%%%%%%%%%%%%%%%%%%%%%%%%%%%%%%%%
\iffalse
\textbf{OLD: Standartox aggregates the test results according to chosen filters in a two step process. Firstly the filtered test results are aggregated by the CAS number, the chosen taxon and the selected test duration. Secondly, the returned data is then aggregated by the CAS number. The former can't be influenced by the user and calculates either the minimum or the median depending on the amount of results to aggregate (n <= 2: minimum, if n > 2: median). Thereof the second step calculates the minimum, the maximum, the median, the geometric mean, or the arithmetic mean as an aggregate.}
\fi


