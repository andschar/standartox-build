
\section*{Background \& Summary [max 700 words]}

A large number of chemicals such as pharmaceuticals, pesticides and synthetic hormones are in daily use all over the world with insufficient knowledge about possible adverse effects on the environment. In Europe alone some 100,000 chemicals are estimated to be in current use, whereof 30,000 are produced in quantities larger than one ton per year \citep{breithaupt_costs_2006}. Although for some chemicals advanced and standardized test methods \citep{oecd_oecd_2018} have been developed, data on toxicity tests for a lot of chemicals, especially newly emerging ones remains scarce and if existent, the information is represented sparsely and in inconsistent formats \citep{gessner_fostering_2016}. Therefore we developed \standardtox{}, a tool to aggregate and standardize toxicity test data from various sources [EPA and ETOX] which provides the information in a reproducible manner.



To our knowledge this ist the first approach

reporducible

time saving





data base that collects, processes and aggregates ecotoxicological tests results in order to subsequently publish it in an harmonized form as a web application - the \etoxbase{} tool: \href{http://139.14.20.252:3838/etox-base-shiny/}. 





Chemicals are brought to the environment deliberately - as it is the case of pesticides, or arrive therein as a byproduct from other processes (e.g. atmospheric emissions or wastewater) \citep{schwarzenbach_challenge_2006}. Ecosystems provide essential services to human societies such as drinking and irrigation water, food and climate regulation. These services are products of different ecosystem functions that crucially depend on the integrity of the populations and communities that drive these ecosystem functions. However, besides habitat degradation, climate change and nutrient enrichment, pollution with man-made chemical toxicants threatens these populations and communities in various ways which are currently not fully understood \citep{steffen_anthropocene_2007}. Pollution with man-made chemical toxicants was indeed identified as one of three major environmental problems for which research gaps hamper the derivation of planetary boundaries, i.e. thresholds beyond which irreversible state shifts may occur \citep{steffen_anthropocene_2007}. Bernhardt et al. \citet{bernhardt_synthetic_2017} argue further that the knowledge gap how chemicals effect populations and communities and hence ecosystem functions and ecosystem services, would also impede society's ability to accomplish the Sustainable Development Goals of the United Nations. According to Breithaupt \citet{breithaupt_costs_2006} less than one percent of chemicals released to the environment are thoroughly tested. 




This should facilitate researchers the access to ecotoxicological test results and support modern chemical risk assessment (CRA) workflows. The \etoxbase{} makes use of ecotoxicological test data, obtained from the ECOTOXicology knowledgebase (ECOTOX) created by the U.S. Environmental Protection Agency (EPA). ECOTOX collects raw, non harmonized ecotoxicological test data on aquatic and terrestrial wildlife as well as plants and publishes it on a quarterly basis. By the time of writing it contained 926,108 test results, comprising 11,685 chemicals tested on 12,668 different species \citep{elonen_ecotoxicology_2018}. The collected ecotoxicological test data contain a variety of measured endpoints such as Effective Concentrations (\ecfifty{}) values, No-observed effect concentrations (NOEC) or lowest observed effect concentrations (LOEC). Likewise the data set contains different effect measures such as mortality, intoxication and growth as well as a multiplicity of test durations ranging from seconds to weeks or even years. The \etoxbase{} aggregates this information according to the user's inputs. These can then be used for the derivation of risk indicators such as Species Sensitivity Distributions (SSDs) \citep{posthuma_species_2002} and Toxic Units (TUs), which represent two prominent concepts to assess effects on organisms in ecotoxicology. The former combines toxicity data of several organism groups towards a chemicals to estimate effects on biotic communities, the latter refers to effects of a specific organism (group) towards a chemical. The two concepts are widely used in ecotoxicology \citep{kefford_definition_2011, schafer_effects_2011} since the allow for a comparison of toxicities across multiple chemicals and biological communities, respectively. Although performed on one and the same organism and chemical, outcomes of different ecotoxicological tests vary greatly due to differences in test parameters, genetic differences in individual populations or other irrepressible factors making a selection of appropriate test results for CRA laborious. The \etoxbase{} aims to relief this process and provide single toxicity values for organism, chemical and test duration combinations. The \etoxbase{} downloads every new version of the ECOTOX data base and performs several quality checks, e.g. unit harmonization, and cleaning steps on it. Hence, new scientific test results are constantly incorporated. Additionally the \etoxbase{} queries chemical- and organism-specific variables from other open data bases to fill information gaps such as water solubility, organism habitat or regional occurrence patterns in the ECOTOX data base.



    \item standardized toxicity values
    
    \item reproducible
    
    \item time saving
    

%%%%%%%%%%%%%%%%%%%%%%%%%%%%%%%%%%%%%%%%%% OLD %%%%%%%%%%%%%%%%%%%%%%%%%%%%%%%%%%%%%%%%%%%

   
\iffalse

A large number of chemicals such as pharmaceuticals, pesticides and synthetic hormones are in daily use all over the world with insufficient knowledge about possible adverse effects on the environment. In Europe alone some 100,000 chemicals are estimated to be in current use, whereof 30,000 are produced in quantities larger than one ton per year \citep{breithaupt_costs_2006}. Chemicals are brought to the environment deliberately - as it is the case of pesticides, or arrive therein as a byproduct from other processes (e.g. atmospheric emissions or wastewater) \citep{schwarzenbach_challenge_2006}. Ecosystems provide essential services to human societies such as drinking and irrigation water, food and climate regulation. These services are products of different ecosystem functions that crucially depend on the integrity of the populations and communities that drive these ecosystem functions. However, besides habitat degradation, climate change and nutrient enrichment, pollution with man-made chemical toxicants threatens these populations and communities in various ways which are currently not fully understood \citep{steffen_anthropocene_2007}. Pollution with man-made chemical toxicants was indeed identified as one of three major environmental problems for which research gaps hamper the derivation of planetary boundaries, i.e. thresholds beyond which irreversible state shifts may occur \citep{steffen_anthropocene_2007}. Bernhardt et al. \citet{bernhardt_synthetic_2017} argue further that the knowledge gap how chemicals effect populations and communities and hence ecosystem functions and ecosystem services, would also impede society's ability to accomplish the Sustainable Development Goals of the United Nations. According to Breithaupt \citet{breithaupt_costs_2006} less than one percent of chemicals released to the environment are thoroughly tested. Although for some chemicals advanced and standardized \citep{oecd_oecd_2018} test methods have been developed, ecotoxicological tests for a lot of chemicals, especially newly emerging ones remain scarce and if existent, the information is represented sparsely and in inconsistent formats. \citep{gessner_fostering_2016}. Therefore we developed a data base that collects, processes and aggregates ecotoxicological tests results in order to subsequently publish it in an harmonized form as a web application - the \etoxbase{} tool: \href{http://139.14.20.252:3838/etox-base-shiny/}. This should facilitate researchers the access to ecotoxicological test results and support modern chemical risk assessment (CRA) workflows. The \etoxbase{} makes use of ecotoxicological test data, obtained from the ECOTOXicology knowledgebase (ECOTOX) created by the U.S. Environmental Protection Agency (EPA). ECOTOX collects raw, non harmonized ecotoxicological test data on aquatic and terrestrial wildlife as well as plants and publishes it on a quarterly basis. By the time of writing it contained 926,108 test results, comprising 11,685 chemicals tested on 12,668 different species \citep{elonen_ecotoxicology_2018}. The collected ecotoxicological test data contain a variety of measured endpoints such as Effective Concentrations (\ecfifty{}) values, No-observed effect concentrations (NOEC) or lowest observed effect concentrations (LOEC). Likewise the data set contains different effect measures such as mortality, intoxication and growth as well as a multiplicity of test durations ranging from seconds to weeks or even years. The \etoxbase{} aggregates this information according to the user's inputs. These can then be used for the derivation of risk indicators such as Species Sensitivity Distributions (SSDs) \citep{posthuma_species_2002} and Toxic Units (TUs), which represent two prominent concepts to assess effects on organisms in ecotoxicology. The former combines toxicity data of several organism groups towards a chemicals to estimate effects on biotic communities, the latter refers to effects of a specific organism (group) towards a chemical. The two concepts are widely used in ecotoxicology \citep{kefford_definition_2011, schafer_effects_2011} since the allow for a comparison of toxicities across multiple chemicals and biological communities, respectively. Although performed on one and the same organism and chemical, outcomes of different ecotoxicological tests vary greatly due to differences in test parameters, genetic differences in individual populations or other irrepressible factors making a selection of appropriate test results for CRA laborious. The \etoxbase{} aims to relief this process and provide single toxicity values for organism, chemical and test duration combinations. The \etoxbase{} downloads every new version of the ECOTOX data base and performs several quality checks, e.g. unit harmonization, and cleaning steps on it. Hence, new scientific test results are constantly incorporated. Additionally the \etoxbase{} queries chemical- and organism-specific variables from other open data bases to fill information gaps such as water solubility, organism habitat or regional occurrence patterns in the ECOTOX data base.

\fi

