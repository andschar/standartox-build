\section{Conclusion}

Due to the steady incorporation of new ecotoxicity data, its values produced by Standartox will be subject to change and most certainly produce slightly different results in its aggregations in the future. However, we regard this as an advantage, since other published works that aim in similar directions often constitute a singular effort or require manual work for each update. Standartox, in contrast automates these processes and as already described would provide means to access older versions anyhow, to assure reproducibility. In comparison to rule-based approaches, such as the PPDB for the derivation of single ecotoxicity values, Standartox has the advantage to not be prone to the subjectivity, the exhaustiveness and the incompleteness of a human induced set of rules. Above all, Standartox provides quick access through its design to be queried via the R language. In general, Standartox is designed to foster chemical risk assessment (CRA). Due to an increased amount of available ecotoxicological test data, it becomes fundamental to provide and distribute ecotoxicity information in adequate formats, both easily accessible for humans and easily processable for machines. Standartox is a tool that does this in a generalizing way and puts its focus on the aggregation of toxicity data, thereby adding a piece to the puzzle of modern ecotoxicological analyses.