\section{Summary}
An increasing number of chemicals such as pharmaceuticals, pesticides and synthetic hormones are in daily use all over the world. In Europe alone, some 100,000 chemicals are estimated to be in current use, whereof 30,000 are produced in quantities larger than one ton per year \citep{breithaupt_costs_2006}. Except for pesticides that are released into the environment deliberately, most chemicals enter the environment as a result of their use through different paths (e.g. atmospheric emission and deposition or discharge through wastewater) \citep{schwarzenbach_challenge_2006}. In the environment, chemicals can adversely affect populations and communities and in turn related ecosystem functions \citep{rhind_anthropogenic_2009, clements_community_2009, hallmann_declines_2014, barracaracciolo_pharmaceuticals_2015, johnston_review_2015, hallmann_more_2017, sanchez-bayo_worldwide_2019}. Ultimately, this may compromise natures contribution to human well-being, for example the ecosystem services clean drinking and irrigation water as well as food production \citep{peters_review_2013, vandersluijs_neonicotinoids_2013, barracaracciolo_pharmaceuticals_2015, yamamuro_neonicotinoids_2019}. Pollution with man-made chemicals has been identified as one of three major environmental problems for which research gaps hamper the derivation of planetary boundaries, i.e. thresholds beyond which irreversible state shifts may occur \citep{steffen_anthropocene_2007, steffen_planetary_2015}. \citet{bernhardt_synthetic_2017} argue that the knowledge gap how chemicals affect populations, communities and in turn ecosystem functions and services, may impede the accomplishment of the Sustainable Development Goals of the United Nations. Even highly regulated chemical compounds such as pesticides have been shown to cause strong adverse effects on organisms, such as birds \citep{hallmann_declines_2014} or fish \citep{yamamuro_neonicotinoids_2019}, questioning the current regulation efforts \citep{schafer_future_2019}. To evaluate the risks from chemicals for ecosystems data on their toxicity, which is typically produced in standardised ecotoxicological laboratory tests is required. For example, \citet{morrissey_neonicotinoid_2015} used ecotoxicological test results from 49 insect and crustacean species to evaluate the effect of neonicotinoids, an insecticide class on the aquatic ecosystem. Also, \citet{malaj_organic_2014} compiled experimental toxicity test results for 223 various chemical compounds to assess the health of freshwater ecosystems in Europe. Similarly, permissible environmental concentrations are often derived from these test data, typically by a combination with safety factors to account for uncertainties, thereby mostly relying on a few, well tested standard organisms, such as the water flea \textit{Daphnia magna}, the fathead minnow \textit{Pimephales promelas}, the microalga \textit{Raphidocelis subcapitata} to name a few. Researchers have conducted experiments on a much greater variety of organisms though. Given that data sets with large numbers of analysed chemicals are becoming increasingly available, ecotoxicity data are pivotal for regulatory risk assessment (RRA) and science. To date, only few initiatives exist that aim to create a public resource of harmonized and quality controlled ecotoxicological data, such as the United States Environmental Protection Agency ECOTOXicology knowledgebase (ECOTOX) (ca. 1,000,000 test results, ca. 13,000 taxa, ca. 12,000 chemicals) \citep{elonen_ecotoxicology_2018}, the German Environmental Agency's Information System Ecotoxicology and Environmental Quality Targets (ETOX) \citep{umweltbundesamt_etox_2019}, the Pesticide Property Data Base (PPDB) (ca. 2000 chemicals) \citep{lewis_international_2016} or the Envirotox database \citep{healthandenvironmentalsciencesinstitutehesi_envirotox_2019, connors_creation_2019}. The former two compile all available test results into a database. However, for many chemicals, multiple ecotoxicity values are available for the same test organism, which can vary strongly and the lack of associated quality information and different units hamper reproducible science. The PPDB database in contrast provides single ecotoxicity values only for pesticides and a few selected test organisms, thereby ignoring vast amounts of data for other species and chemicals. Envirotox puts its focus on aquatic organisms. Moreover, data analyses often require links to additional data resources, for example to append additional chemical and species information (e.g. chemical properties, habitat of species), which calls for more automated procedures. We therefore developed Standartox, a tool and database that aims to overcome the limitations of other databases by continuously incorporating the ever-growing number of test results in an automated process workflow that ultimately leads to a cleaned and harmonised ecotoxicity collection and single aggregated ecotoxicity values for a specific chemical-organism test combination. Standartox makes use of the publicly available and quarterly released ECOTOX database \citep{usepa_ecotox_2019} and restricts the data to three commonly used endpoints in ecotoxicology (i.e. XX50, LOEX \& NOEX), leading to about 600,000 ecotoxicological test results including 8000 chemicals, tested on about 8000 taxa. Standartox allows users to filter test results according to several parameters, e.g. refining a search for ecotoxicity data on organisms occurring in specific habitats or regions of the world. Above that, Standartox aggregates ecotoxicological test results in a standardized way, by calculating the minimum, the geometric mean and the maximum of the result for each chemical and the associated, user-defined test parameters. This relaxes the commonly occurring problems of the variability of multiple ecotoxicological test results and which values to choose for chemical risk assessment (CRA). Hence, Standartox provides the basis for reproducible science and combines information from different sources to simplify the derivation of more advanced risk indicators such as Species Sensitivity Distributions (SSD) and Toxic Units (TU), which represent two prominent concepts to assess effects on organisms in ecotoxicology \citep{posthuma_species_2002, kefford_definition_2011, schafer_effects_2011}. Besides aggregating ecotoxicological test results, Standartox provides a concise overview of the status of tested chemicals, allowing the identification of potential research gaps. Moreover, Standartox could help in reducing the millions of animals used for toxicity testing each year \citep{hartung_chemical_2009}. Standartox comes with two front-ends, a web application (APP): \url{standartox.uni-landau.de} and a R \citep{rcoreteam_language_2017} package \textit{standartox}, providing convenience structures and thereby largely reducing processing time for users.

