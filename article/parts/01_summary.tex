\section{Summary}
An increasing number of chemicals such as pharmaceuticals, pesticides and synthetic hormones are in daily use all over the world. In Europe alone, some 100,000 chemicals are estimated to be in current use, whereof 30,000 are produced in quantities larger than one ton per year \citep{breithaupt_costs_2006}. Except for pesticides that are released into the environment deliberately, most chemicals enter the environment as a result of their use through different paths (e.g. atmospheric emission and deposition or discharge through wastewater) \citep{schwarzenbach_challenge_2006}. In the environment, chemicals can adversely affect populations and communities and in turn related ecosystem functions \citep{schafer_thresholds_2012, malaj_organic_2014, hallmann_declines_2014, barracaracciolo_pharmaceuticals_2015, johnston_review_2015}. Ultimately, this may compromise natures contribution to human well-being, for example the ecosystem services clean drinking and irrigation water as well as food production \citep{peters_review_2013, vandersluijs_neonicotinoids_2013, yamamuro_neonicotinoids_2019}. 
Pollution with man-made chemicals has been identified as one of three major environmental problems for which research gaps hamper the derivation of planetary boundaries, i.e. thresholds beyond which irreversible state shifts may occur \citep{steffen_anthropocene_2007, steffen_planetary_2015}. \citet{bernhardt_synthetic_2017} argue that the knowledge gap how chemicals affect populations, communities and in turn ecosystem functions and services, may impede the accomplishment of the Sustainable Development Goals \citep{rosa_transforming_2017} of the United Nations. Even highly regulated chemical compounds such as pesticides have been shown to cause strong adverse effects on non-target organisms, such as birds \citep{hallmann_declines_2014} aquatic insects \citep{beketov_pesticides_2013} or fish \citep{yamamuro_neonicotinoids_2019}, questioning the current regulation efforts \citep{schafer_future_2019}.

To evaluate the risks from chemicals for ecosystems, data on their toxicity is required, which is typically produced in standardised ecotoxicological laboratory tests. For example, \citet{morrissey_neonicotinoid_2015} used ecotoxicological test results from 49 insects and crustaceans to evaluate the effect of neonicotinoid insecticides in the aquatic ecosystem. Also, \citet{malaj_organic_2014} compiled experimental toxicity test results for 223 chemicals to assess the health of freshwater ecosystems in Europe. Similarly, permissible environmental concentrations are often derived from these test data, typically by a combination with safety factors to account for uncertainties. The test data mainly relate to a few, well tested standard organisms, such as the brown rat \textit{Rattus norvegicus}, the water flea \textit{Daphnia magna} and the microalga \textit{Raphidocelis subcapitata}. Nevertheless, a much greater variety of organisms has been used in ecotoxicological experiments.

To date, only few initiatives exist that aim to create a public resource of ecotoxicological data, such as the United States Environmental Protection Agency ECOTOXicology knowledgebase (ECOTOX) (ca. 1,000,000 test results, 13,000 taxa, 12,000 chemicals) \citep{elonen_ecotoxicology_2018}, the German Environmental Agency's Information System Ecotoxicology and Environmental Quality Targets (ETOX) \citep{umweltbundesamt_etox_2019}, the Pesticide Properties Data Base (PPDB) (ca. 2000 pesticides) \citep{lewis_international_2016} or the Envirotox database \citep{healthandenvironmentalsciencesinstitutehesi_envirotox_2019, connors_creation_2019}. The former two compile all available results from experiments into a database. However, for many chemicals, multiple ecotoxicity values are available for the same test organism, which can vary strongly and the lack of associated quality information and heterogeneous units hamper reproducible science. The PPDB database, in contrast, provides single ecotoxicity values only for pesticides and a few selected test organisms, thereby covering only a minor fraction of the vast amount of ecotoxicological data. The Envirotox is limited to aquatic organisms. Moreover, data analyses often require links to additional data resources, for example to append additional chemical and species information (e.g. chemical properties, habitat of species), which calls for more automated procedures. 

We therefore developed Standartox, a tool and database that aims to overcome the limitations of other databases by continuously incorporating the ever-growing number of test results in an automated process workflow that ultimately leads to a cleaned and harmonised ecotoxicity data collection and provides methods to derive single aggregated ecotoxicity values for a specific chemical-organism test combination. Standartox makes use of the publicly available and quarterly updated ECOTOX database \citep{usepa_ecotox_2019} and restricts the data to commonly used endpoints in ecotoxicology, such as half maximal effective concentration (EC\textsubscript{50}) or no-observed-adverse-effect concentrations (NOEC), leading to about 600,000 ecotoxicological test results including 8000 chemicals, tested on about 8000 taxa. Users can filter test results according to several parameters, e.g. refining a search for ecotoxicity data on organisms occurring in specific habitats or regions of the world. Above that, Standartox aggregates ecotoxicological test results in a standardized way, by calculating the minimum, the geometric mean and the maximum of the results for each chemical and the associated, user-defined test parameters. This tackles the commonly occurring problem of the variability of multiple ecotoxicological test results and which values to choose for assessing the effects of chemicals. Hence, Standartox provides the basis for reproducible science and combines information from different sources to simplify the derivation of risk indicators such as Species Sensitivity Distributions (SSD) and Toxic Units (TU), which represent two prominent concepts to assess effects on organisms in ecotoxicology \citep{posthuma_species_2002, kefford_definition_2011, schafer_effects_2011}. Besides aggregating ecotoxicological test results, Standartox provides a concise overview of the tested chemicals, allowing the identification of potential knowledge gaps. Moreover, Standartox could help in reducing the millions of animals used for toxicity testing each year due to facilitated ecotoxicity data access \citep{hartung_chemical_2009}. Standartox comes with two front-ends, a web application: \url{standartox.uni-landau.de} and a R \citep{rcoreteam_language_2017} package \textit{standartox}, providing convenience structures and thereby largely reducing processing time for users.

