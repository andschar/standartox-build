\section{Methods}
An automated processing pipeline downloads the quarterly released ECOTOX database, performs several preparation steps on it and exports a final Standartox data set. This data set is accessible via a web application (APP) and an application programming interface (API). An API provides the means for machine communication between a host and a client and thus allowing scriptable data queries. To facilitate the API access, the R \citep{rcoreteam_language_2017} package \textit{standartox} is built.

\subsection{Processing}
Standartox downloads the quarterly released EPA ECOTOX database whenever a new version is published and builds it into a local PostgreSQL database \citep{szocs_build_2019}. Subsequently Structured Query Language (SQL) functions for further processing the data are developed. In addition lookup tables (Supplement \ref{sup:conv-concentration}, \ref{sup:conv-duration}) that enable duration and concentration unit conversions of the data are created. A meta table providing information such as the release version of the ECOTOX database is then added. In the next step, Chemical Abstracts Service Numbers (CAS) and International Chemical Identifier (InChI) as well as taxonomic names are used to query additional information from publicly available databases on chemicals and organisms, respectively. This includes the Compendium of Pesticide Common Names \citep{wood_compendium_2019}, the Chemical Entities of Biological Interest (ChEBI) database \citep{hastings_chebi_2016}, the Chemical Identifier Resolver (CIR) service \citep{nationalinstitutesofhealthnih_chemical_2019}, the PHYSPROP database, the Pubchem database \citep{kim_pubchem_2016}, Eurostat \citep{europeancommission_eurostat_2019} and Wikidata \citep{vrandecic_wikidata_2014} for chemicals and the World Register of Marine Species (WoRMS) \citep{wormseditorialboard_world_2018} and the Global Biodiversity Information Facility (GBIF) \citep{_gbif_2019} (Table \ref{tab:data-base-additional}) for habitat and spatial distribution of organisms. This information is added to Standartox to allow filtering for specific classes of chemicals as well as spatial distribution (i.e. continents) and habitat preferences (e.g. freshwater) of individual taxa. Taxa that where not identified to at least genus level are excluded. Finally, the Standartox data set is compiled, which includes the harmonisation of data, e.g. through conversion of units related to the concentrations and test duration. Out of the unique 1229 concentration units in the ECOTOX database, Standartox retains only those (n = X) that are unambiguously convertible to one of the following units: $\mu$g/l, g/m2, ppb, mg/kg, \% and flags, the remaining as \textit{other}, resulting in 873473 out of 952625 test results. Furthermore, the units are cleaned, for example through removing additional information in the field such as \textit{food}, \textit{soil}, \textit{ai} that are also coded in other variables and hinder the processing of units. Concentrations that are given as rates such as per day (mg/kg/day) are generally excluded.
Likewise, out of the 125 test duration units in the EPA ECOTOX, Standartox retains only those (n = X) that can be converted to hours, excluding ambiguous duration units such as harvest or lifetime and keeping 905110 out of 952625 test results. Thereby concentration unit additions such as \textit{food}, \textit{soil}, \textit{ai (active ingredient)} etc. are neglected since they are also coded in other variables and only hinder the processing of units. Units such as per day (mg/kg/day) are generally excluded. Test endpoints are restricted to three groups, named NOEX, LOEX and XX50. The former two represent no-observed-adverse-effect and lowest-observed-adverse-effect concentrations or levels, respectively. The latter includes various sub-groups of the half maximal effective concentration, where half of the tested individuals show an effect (e.g. EC50, LC50, LD50), which is a common measure in ecotoxicology \citep{malaj_organic_2014}. Other endpoints, such as Bioconcentration factors (BCF), non-half maximal effective concentrations (e.g. IC10, EC25, LD99) or maximum acceptable toxicant (MATC) concentrations are removed. Beyond that, filters for the CAS number, the concentration type (e.g. active ingredient, formulation), the chemical class (e.g. fungicides, metals), taxon, organism habitat (e.g freshwater, terrestrial) and region (e.g. Europe, Asia), test duration as well as effect type (e.g. mortality, growth) are created. In order to detect possible outliers among the test results, we flag values that exceed 1.5 times the interquartile range (IQR) within groups with identical test parameters (e.g. chemical, taxon, duraiton). Along with the compiled Standartox data set, a catalog, listing all distinct entries and value ranges, for categorical and continuous variables, respectively is created. Finally, we run quality control scripts that check the accuracy of the data. Single processing scripts are listed in Figure \ref{fig:pipeline-tree}. The final Standartox table together with the catalog is exported and accessible via web application and the API.

\begin{figure}
    \includestandalone[scale=0.5]{article/tikz/pipeline-organigram}
    %\input[scale = 0.5]{article/tikz/pipeline-organigram.tex}
    \caption{Organigram of Standartox.}
    \label{fig:stx-organigram}
\end{figure}

\subsection{Access}
The APP and the API load the compressed serialized Standartox data into memory and allow a client to interact with it. The client thereby calls the functions stx\_filter() and stx\_aggregate() that filter and aggregate the data according to specific parameters (Table \ref{tab:app-parameters}). The interactive APP is built in R using the shiny web application framework, which runs with the help of a shiny server \citep{chang_shiny_2018} on \url{standartox.uni-landau.de}. The API is built by using the R package \textit{plumber} \citep{trestletechnologyllc_plumber_2018}, which allows for the creation of Representational state transfer (REST) APIs from R. REST is a software standard that defines web service communication rules. The API is reachable via the URL \url{http://139.14.20.252} and port 8000. Four API-endpoints (\textit{/catalog}, \textit{/filter}, \textit{/aggregate} and \textit{/meta}) can be queried from the client side. The \textit{/catalog} API-endpoint returns a JavaScript Object Notation (JSON) file containing a catalog of possible filter parameters to choose from. The \textit{/filter} returns the filtered Standartox table as a compressed serialized binary file created by the R \textit{fst} package \citep{klik_fst_2019}, to reduce size and allow for fast user queries. The \textit{/aggregate} API-endpoint returns an R function allowing the client to aggregate the filtered data. Lastly, the \textit{/meta} API-endpoint returns a JSON file with meta information, such as the timestamp of the request and the used Standartox version. The API is designed to be used with the R-package \textit{standartox} and therefore uses serialization methods specific to R (rds() from the R package base and fst() from the package fst). In order to facilitate the API usage the R-package \textit{standartox} is created.

\subsection{Technical Validity}
To guarantee appropriate unit conversion and harmonisation, we compare for each of the 1229 distinct concentration and for each of the 129 duration units the automatically converted units to one manually calculated one.

