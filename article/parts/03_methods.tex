%%The Methods should include detailed text describing any steps or procedures 
%%used in producing the data, including full descriptions of the experimental 
%%design, data acquisition assays, and any computational processing (e.g. 
%%normalization, image feature extraction). Related methods should be grouped 
%%under corresponding subheadings where possible, and methods should be described 
%%in enough detail to allow other researchers to interpret and repeat, if required, 
%%the full study. Specific data outputs should be explicitly referenced via data 
%%citation (see Data Records and Data Citations, below). Authors should cite 
%%previous descriptions of the methods under use, but ideally the method 
%%descriptions should be complete enough for others to understand and reproduce 
%%the methods and processing steps without referring to associated publications. 
%%There is no limit to the length of the Methods section.
\section*{Methods}

Standartox consists of three parts of software that are combined to .... Firstly a data acquisition and preparation part that downloads the quarterly released new versions of the EPA ECOTOX data base and automatically builds Standartox based on several processing steps on the data. Secondly, the final Standartox table is accessible via a web application (APP) and an application programming interface (API). Lastly, the R \citep{rcoreteam_language_2017} package standartox constitutes an easily usable way of accessing the tool via the API.

\subsection*{Processing pipe line}
The US Environmental Protection Agency (EPA) releases the EPA ECOTOX data base on a quarterly basis. The latest version, released on 12.09.2019 contains data on 11,756 chemicals, 12,906 taxa, 49,153 references and 952,634 results \citep{usepa_ecotox_2019} constituting a vast collection of ecotoxicological tests, which can be downloaded as pipe delimited ASCII files. Standartox downloads the EPA ECOTOX data base whenever a new version is published and builds into a local PostgreSQL data base \citep{szocs_build_2019}. Subsequently SQL functions for further processing are added and known errors are corrected (and also reported). In addition lookup tables (Supplement \ref{sup:conv-concentration}, \ref{sup:conv-duration}) are created that support filtering and converting the data as well as a meta table providing information such as the release version of the EPA ECOTOX data base. In the next step, Chemical Abstracts Service (CAS) identifiers are used to query identifier data bases, including the Chemical Identifier Resolver (CIR) service \citep{nationalinstitutesofhealthnih_chemical_2019} and the Pubchem data base \citep{kim_pubchem_2016}. Chemical identifiers (CAS, InchI) and taxonomic names are then used to query additional information such as chemical grouping or organism habitat preferences and global occurrence patterns from data bases (Table \ref{tab:data-base-additional}) and include them in the Standardtox database.

\begin{table}[ht]
  \caption{Table of queried data bases}
  \label{tab:data-base-additional}
  \centering
\begin{tabular}{ll}
  \hline
  Data Base & URL \\ 
  \hline
  Chemical Entities of Biological Interest (ChEBI) & https://www.ebi.ac.uk/chebi \\ 
  Chemical Identifier Resolver & https://cactus.nci.nih.gov/chemical/structure \\
  ChemSpider & http://www.chemspider.com \\
  Eurostat & https://ec.europa.eu/eurostat/home \\
  PHYSPROP Database & https://www.srcinc.com/what-we-do/environmental/scientific-databases.html \\
  PubChem & https://pubchem.ncbi.nlm.nih.gov \\ 
  Wikidata & https://www.wikidata.org/wiki/Wikidata:Main_Page \\
  Global Biodiversity Information Facility (GBIF) & https://www.gbif.org \\
  World Register of Marine Species (WoRMS) & http://marinespecies.org \\
  \hline
\end{tabular}
\end{table}


Finally, the Standartox data set is compiled, which includes the harmonisation of data, e.g. through conversion of units related to the concentrations and test duration. Out of the unique 1229 concentration units in the EPA ECOTOX data base, Standartox retains only those (n = X) that are unambiguously convertible to one of the following units: ug/l, g/m2, ppb, mg/kg, \% and flags, the remaining as \textit{other}, resulting in 873473 out of 952625 test results. Furthermore, the units are cleaned, for example through removing additional information in the field such as \textit{food}, \textit{soil}, \textit{ai} that are also coded in other variables and hinder the processing of units. Concentrations that are given as rates such as per day (mg/kg/day) are generally excluded.
Likewise, out of the 125 test duration units in the EPA ECOTOX, Standartox retains only those (n = X) that can be converted to hours, keeping 905110 out of 952625 test results). Thereby concentration unit additions such as \textit{food}, \textit{soil}, \textit{ai} etc. are neglected since they are also coded in other variables and only hinder the processing of units. Units such as per day (mg/kg/day) are generally excluded. Test endpoints are refined to three endpoint groups NOEX, LOEX and XX50. The former two represent no-observed-adverse-effect levels and lowest-observed-adverse-effect levels, respectively. The latter includes various versions of halft of the maximal effective concentration (e.g. EC50, LC50, LD50 etc.) commonly used in ecotoxicology, e.g. the concentration where half of the test individuals show an effect. Other endpoints are removed. Beyond that, filters for the CAS number, the concentration type (e.g. active ingredient, formulation), the chemical class (e.g. fungicides, metals), taxon, organism habitat and region, test duration as well as effect type (e.g. mortality, growth) (See Table \ref{tab:app-parameters}) are created. Along with the compiled Standartox data set, a catalog, listing all distinct entries and value ranges, for categorical and continuous variables, respectively is created. Finally, we run quality control scripts that check the accuracy of the data. Single processing scripts are listed in Figure \ref{fig:pipeline-tree}. The final Standartox table together with the catalog is exported and accessible via web application and the API.

\begin{figure}
    \includestandalone[scale=0.5]{article/tikz/pipeline-organigram}
    %\input[scale = 0.5]{article/tikz/pipeline-organigram.tex}
    \caption{Organigram of Standartox.}
    \label{fig:stx-organigram}
\end{figure}

\subsection*{Web application (APP) and Application Programming Interface (API)}
The web application (APP) and the Application Programming Interface (API) load the compressed serialized Standartox data into memory and allow a client to interact with it. The client can then call the functions stx\_filter() and stx\_aggregate() that filter and aggregate the data according to specific parameters (Table \ref{tab:app-parameters}).

\subsubsection*{Web application (APP)}
The interactive APP was built in R using the shiny web application framework, which runs with the help of a shiny server \citep{chang_shiny_2018} on \url{www.standartox.uni-landau.de}. It is a set of R functions that allow a user to filter, aggregate and download the data.

\subsubsection*{Application programming interface (API)}
The API is built by using the R package plumber \citep{trestletechnologyllc_plumber_2018}, which allows for the creation of RESTful APIs from R. The API is reachable via the URL \url{http://139.14.20.252} and port 8000. Four API-endpoints (\textit{/catalog}, \textit{/filter}, \textit{/aggregate} and \textit{/meta}) can be queried from the client side. The \textit{/catalog} API-endpoint returns a JavaScript Object Notation (JSON) file containing a catalog of possible filter parameters to choose from. The \textit{/filter} returns the filtered Standartox table as a compressed serialized binary file created by the R fst package \citep{klik_fst_2019}, to reduce size and allow for fast user queries. The \textit{/aggregate} API-endpoint returns an R function allowing the client to aggregate the filtered data. Lastly, the \textit{/meta} API-endpoint returns a JSON file with meta information, such as the timestamp of the request and the used Standartox version. The API is designed to be used with the R-package standartox and therefore uses serialization methods specific to R (rds() from the R package base and fst() from the package fst).

\subsubsection*{R-package}
The R-package standartox accesses the API through two functions \textit{stx\_catalog()} and \textit{stx\_query()}. The former accesses the \textit{/catalog} API-endpoint and returns a R list object of possible filter parameters (Table \ref{tab:app-parameters}) that can be used in \textit{stx\_query()}. The latter returns a R list of three tables (i.e. R data.frames) containing the filtered data set, the aggregated data set and a table with the meta information retrieved from the API endpoints \textit{/filter} and \textit{/meta}.

\subsection*{Technical Validity}
To guarantee appropriate conversion and harmonisation of units we take a random sample of the 50 most occurring concentration and six of the most occurring duration units and compare units created by Standartox to manually calculated unit conversions. This is done at the end of the processing pipeline and stops the it if not all units match.

%%%%%%%%%%%%%%%%%%%%%%%%%%%%%%%%%%%%%%%%%% TODO %%%%%%%%%%%%%%%%%%%%%%%%%%%%%%%%%%%%%%%%%%
\iffalse
- Endpoint: For the last parameter here, only be exclusive. 

TODO: How does this fit?
\begin{table}
    
\begin{tabular}{l|l|r}
\hline
Step & Description & Number of records\\
\hline
all & All EPA ECOTOX values & 952634\\
\hline
cleaned & Cleaned data set & 952625\\
\hline
effect & Removes data without an Effect (e.g. Mortality, Growth, etc.) & 922575\\
\hline
endpoint & Final harmonised data set & 268785\\
\hline
\end{tabular}

    \caption{Reduction of data in the compilation process for Standartox}
    \label{tab:data-refinement}
\end{table}

\fi


%%%%%%%%%%%%%%%%%%%%%%%%%%%%%%%%%%%%%%%%%% OLD %%%%%%%%%%%%%%%%%%%%%%%%%%%%%%%%%%%%%%%%%%%
\iffalse
\textbf{OLD: Standartox aggregates the test results according to chosen filters in a two step process. Firstly the filtered test results are aggregated by the CAS number, the chosen taxon and the selected test duration. Secondly, the returned data is then aggregated by the CAS number. The former can't be influenced by the user and calculates either the minimum or the median depending on the amount of results to aggregate (n <= 2: minimum, if n > 2: median). Thereof the second step calculates the minimum, the maximum, the median, the geometric mean, or the arithmetic mean as an aggregate.}
\fi


